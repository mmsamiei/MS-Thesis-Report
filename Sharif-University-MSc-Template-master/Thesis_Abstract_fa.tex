\thispagestyle{empty}
\centerline{\textbf{\large{چکیده}}}
\begin{quote}
خلق ماشینی که قادر به گفتگوی هوشمند با انسان از طریق بستر زبان طبیعی باشد، یکی از آرزوهای اولیه و در عین حال از اهداف 
نهایی و غایی هوش مصنوعی به شمار می‌رود. 
در سال‌های اخیر به لطف رشد روزافزون یادگیری ژرف و تولیدشدن و در دسترس بودن دادگان عظیم، مسئله سامانه‌ مکالمه (گپ‌زن) مورد توجه
پژوهشگران واقع شده و رشد قابل توجهی را تجربه کرده است.
علی رغم این پیشرفت های چشم‌گیر، اما این مکالمه‌گر‌های داده محور اغلب قادر به ارائه دانش حاصل از دنیای واقعی و محتوامحور در بستر مکالمات خود نیستند که این پدیده باعث واردشدن خدشه به جنبه هوشمندی آن‌ها می‌باشد و استفاده از آن‌ها را در کاربردهای واقعی
و دامنه‌باز
 با مشکل روبرو می‌کند.
از جمله نواقصی که باعث بروز این مشکل می‌شود می‌توان به نبود مکالمات دانش‌محور کافی در دادگان آموزش و همچنین لحاظ نکردن دانش خارجی در معماری شبکه‌های عمیق طراحی‌شده موجود، اشاره کرد.
\newline  
هدف از این پژوهش ارائه مدلی با استفاده از یادگیری ژرف جهت مکالمه مبتنی بر محتوای دانش و واقعیات بیرونی است؛ به ترتیبی که سامانه با استفاده از منابع دانش خارجی و قابل به روزرسانی قادر می‌شود که گفتگویی غنی از اطلاعات موجود در دانش بیرونی را انجام دهد. ذکر این نکته ضروری است که منابع دانش خارجی در این پژوهش به صورت مجموعه‌ای از اسناد متنی در نظر گرفته شده‌اند.
\newline
از آن جایی که در سالیان اخیر مدل‌های برآمده شبکه‌های از پیش آموزش داده شده توانسته‌اند پرچمدار مسائل مختلف در حوزه پردازش زبان باشند، در این تحقیق نیز سعی شده است تا رویکرد حل مسئله مبتنی بر استفاده از قدرت این شبکه‌ها و یادگیری انتقالی باشد. آزمایش‌های این پژوهش نشان‌دهنده برتری روش‌های ارائه شده متکی بر شبکه‌های از پیش آموزش داده شده در مقابل روش‌های پیشین است.

\vskip 1cm
\textbf{کلمات کلیدی:} \textiranic{
یادگیری ژرف،
سامانه مکالمه،
روبات‌ گپ‌زن دامنه‌باز،
روبات گپ‌زن مبتنی بر دانش،
شبکه‌های از پیش آموزش دیده
}
\end{quote}