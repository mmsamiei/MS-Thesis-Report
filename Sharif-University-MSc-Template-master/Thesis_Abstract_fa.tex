\thispagestyle{empty}
\centerline{\textbf{\large{چکیده}}}
\begin{quote}
دنباله‌ها بخش زیادی از اطلاعات دنیای واقعی را تشکیل می‌دهند، که بارزترین نمونه‌ی آن زبان‌های طبیعی است. بسیاری از ساختارهای مهم نیز قابلیت مدل شدن به عنوان دنباله را داشته و داده‌هایی مثل ساختار مولکول، گراف و نت‌های موسیقی را نیز می‌توان به عنوان دنباله در نظر گرفت. از سوی دیگر تولید دنباله‌های جدید و بامعنی در هر حوزه، موضوعی با اهمیت و کاربردی است، مثلا در ترجمه‌ی زبان طبیعی یا کشف ساختار داروی جدید با ویژگی‌های خاص، مساله‌ی تولید دنباله وجود دارد.
با این وجود مشکلات مهم زیادی برای حل مساله‌ی تولید دنباله مطرح است، از جمله این مشکلات می‌توان به گسسته بودن جنس داده‌ها و انتخاب تابع هدف مناسب اشاره کرد. 
روش‌های پایه‌ی ارائه‌شده در این حوزه، دارای مشکلاتی نظیر اُریبی مواجهه میان زمان آموزش و آزمون و تابع هدف نامناسب هستند، از این رو نیاز به روش‌های جدید در این حوزه احساس می‌شود.
\newline  
در چند سال اخیر پیشرفت‌های زیادی در حوزه‌ی تولید تصویر  به وسیله‌ی شبکه‌های مولد مقابله‌ای انجام شده است. همین موضوع باعث شده که استفاده از شبکه‌های مولد مقابله‌ای در تولید دنباله‌ها نیز به تازگی مورد توجه قرار گیرد.  اما به دلیل گسسته بودن جنس دنباله‌ها، این امر به سادگی میسر نبوده و برای حل آن نیاز به استفاده از راهکار‌هایی مثل یادگیری تقویتی و استفاده از تقریب وجود دارد.
به علاوه ناپایداری شبکه‌های مولد مقابله‌ای باعث ایجاد چالش‌های جدید و زیاد شدن پیچیدگی مساله می‌شود.
\newline
در این پژوهش، با بیان رویکردی جدید و مبتنی بر ایده‌ی شبکه‌های مولد مقابله‌ای، به ارائه‌ی روشی برای حل مساله‌ی تولید دنباله با رویکردی تکرار شونده می‌پردازیم، به طوری که مدل در هر گام الگوریتم، با آموزش در مقابل نمونه‌های تولیدی خودش بهبود می‌یابد.
اساس روش پیشنهادی تخمین نسبت چگالی احتمال بوده و با این رویکرد روشی بدون مشکل دربرابر گسستگی دنباله‌ها ارائه شده است.
راهکار ارا‌‌ئه شده نسبت به روش‌های شبکه‌های مولد مقابله‌ای در حوزه‌ی دنباله، آموزشی پایدار‌تر دارد؛ هم‌چنین مشکل اُریبی مواجهه نیز در روش پیشنهادی وجود ندارد.
\newline
از آنجا که ارزیابی مدل‌های مولد خود چالشی مورد تحقیق است، در بخش دیگری از پایان‌نامه به بررسی معیارهای موجود پرداخته و با ارائه سه نحوه‌ی ارزیابی جدید، سعی در رفع مشکل معیار‌های موجود و بهره بردن از نتایج پژوهش‌های مربوط به حوزه‌ی تصویر داریم. روش‌های ارزیابی پیشنهادی برخلاف معیار‌های پیشین که فقط کیفیت نمونه‌ها را بررسی می‌کنند، همزمان به تنوع نمونه‌های تولیدی در کنار کیفیت اهمیت می‌دهند.
آزمایش‌های این پژوهش نشان‌دهنده‌ی برتری روش پیشنهادی در مقابل روش‌های پیشین است.


\vskip 1cm
\textbf{کلمات کلیدی:} \textiranic{
شبکه‌های مقابله‌ای، ایجاد دنباله، شبکه‌های عصبی، یادگیری عمیق
}
\end{quote}