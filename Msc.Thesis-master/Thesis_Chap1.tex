\chapter{مقدمه}\label{Chap:Chap1}
\minitoc

خلق ماشینی که بتواند با انسان از طریق زبان طبیعی به گفتگو بپردازد یکی از آرزو‌های اولیه و در عین حال از اهداف نهایی و غایی هوش مصنوعی می‌باشد.
این ماشین بایستی بتواند ویژگی‌های گفتگوی انسانی را به خوبی تقلید کند و همچنین قادر باشد که در طیف موضوعی وسیعی که می‌تواند از راجع به تئوری نسبیت انیشتین تا آخرین اخبار اقتصادی روز گسترده‌شده باشد صحبت کند.
همچنین این ماشین هوشمند می‌تواند این امنیاز اضافه را نیز داشته باشد از طریق گفتگو با انسان قادر به فهمیدن وظایف مختلف اطلاعاتی (برای مثال رزرو یک فیلم در یک سینما) و انجام آن ها برای انسان باشد.
نیل به این هدف در نهایت می‌تواند موجب پیدایش سامانه‌های هوشمندی نظیر
دستیار مجازی، سامانه توصیه‌گر و ربات گپ‌زن شود که همگی از طریق 
بستر زبان طبیعی سعی در ارتباط با عامل انسانی پیش‌ روی خود دارند.

برای نزدیک شدن به این مهم ماشین‌ها بایستی بتوانند تعدادی توانایی کلیدی را کسب کنند. از جمله این توانایی ها می‌توان به درک زبان طبیعی، به کار بستن حافظه برای حفظ و به یادآوری مفاهیم و دانش در قالب زبان طبیعی،‌ استنتاج در مورد مفاهیم و در نهایت توانایی تولید پاسخ مناسب (چه از نظر صحت زبانی و چه از نظر ارتباط با محتوای گفتگو) اشاره کرد.

با وجود پیشرفت‌های گسترده و چشم‌گیر هوش مصنوعی در سالیان اخیر و ظهور و تکامل یادگیری ماشین و یادگیری ژرف اما تلاش‌ها برای خلق یک سیستم مکالمه‌گر با ویژگی‌های مذکور تا به امروز ناکام مانده‌اند و این مساله یک مساله باز و حل نشده باقی مانده است.
وظیفه مکالمه را در واقع می‌توان به نوعی محل تلاقی سه حوزه تحقیقاتی
هوش مصنوعی، پردازش زبان و بازیابی اطلاعات دانست.
هم اکنون توجه زیادی از سمت محققین این سه حوزه به مسئله مکالمه معطوف شده است و هر پیشرفتی در روند این مساله می‌تواند بیانگر پیشرفتی عظیم در این سه حوزه باشد.

علاوه بر مطالب بالا می‌توان به طور مخصوص از زاویه دید حوزه پردازش زبان نیز وظیفه مکالمه را بررسی کرد. عمده وظایف موجود در حوزه پردازش زبان طبیعی را می‌توان در دو دسته فهم و درک زبان طبیعی و تولید زبان طبیعی گروه‌بندی کرد. اهمیت وظیفه مکالمه در حوزه پردازش طبیعی در این است که تحقق آن در واقع نشان از تحقق در هر دو حوزه فهم زبان طبیعی و تولید زبان طبیعی دارد و در نتیجه وظیفه مکالمه را می‌توان به نوعی مادر تمام وظایف پردازش زبان در نظر گرفت. یکی از اهمیت‌های این موضوع می‌تواند این باشد که همان‌گونه که امروزه مدل‌های از پیش آموزش داده شده با استفاده از مقدار زیادی داده روی وظیفه مدل زبانی آموزش می‌بینند و سپس با کمک انتقال یادگیری روی وظایف دیگری تنظیم می‌شوند، شاید بتوان در آینده همین رویکرد را به جای استفاده از مدل‌های زبانی با مدل‌های مخصوص مکالمه دنبال کرد.



\section{تعریف مساله} \label{chap1:prob_define}
سامانه‌های مکالمه را از منظر مسئله‌ ی مورد هدف گذاری شده‌شان می‌توان به سه دسته تقسیم کرد. این سه دسته مسئله پرسش و پاسخ، تکمیل وظیفه و گپ‌زنی 
هستند.

در مساله پرسش و پاسخ، معمولا با سه عنصر کاربر انسانی، عامل هوشمند و یک منبع دانش مواجه هستیم. منبع دانش در واقع می‌تواند یا به صورت مجموعه‌ای از داده‌های بدون ساختار مانند اسناد متنی یا به صورت یک پایگاه دانش ساختار یافته مانند گراف دانش باشد. روند این مساله به این صورت است که کاربر انسانی یک پرسش را در بستر زبان طبیعی مطرح می‌کند و عامل هوشمند بایستی بتواند با توجه به سوال مفاهیم مربوط را از پایگاه دانش استخراج کرده و گاها روی آن‌ها عملیات استنتاج انجام دهد. پس از استخراج و استنتاج این مفاهیم سپس عامل هوشمند بایستی پاسخ مربوط را در بستر زبان طبیعی تولید کرده و به کاربر انسانی انتقال دهد.
عامل‌های هوشمند مخصوص مسائل پرسش و پاسخ معمولا فاقد حافظه برای مکالمه هستند و هر دور مکالمه آن‌ها با کاربر معمولا تنها یک رفت و برگشت طول می‌کشد. با توجه به کاربرد‌های گسترده علمی و صنعتی که عامل‌های هوشمند پرسش و پاسخ می‌توانند داشته باشند، از سال های قبل توجه و انرژی بسیاری از محققین روی این مساله جذب شده است و اکنون تعداد زیادی دادگان برای مسائل با تنظیمات گوناگون پرسش و پاسخ طراحی و جمع‌آوری شده اند. از مشهورترین این دادگان می‌توان به دادگان اسکواد، نچرال کوئسچن و سی ان ان  و ویکی کیو ای اشاره کرد.

در مقابل دسته بعدی سامانه‌های مکالمه‌گر،‌ مکالمه‌گر‌های وظیفه هستند که در پی این هستند تا با استفاده از زبان طبیعی یک ارتباط موثر بین خود و عامل انسانی فراهم کنند. این ارتباط در نهایت موجب بهبود درک متقابل میان این دو شده و موجب می‌شود تا وظیفه تعیین شده
برای این سامانه به خوبی انجام شود.
این وظیفه می‌تواند برای مثال رزرو بلیت سینما، تنظیم کردن یاد‌آور زمانی،‌خرید از فروشگاه اینترنتی یا انواع وظایف دیگر باشد.

\section{مقایسه مدل‌های زبانی با فضای نهان و بدون فضای نهان} \label{chap1:latent_or_not}

\section{رویکردهای آموزشی} 
\label{chap1:sec:approaches}


\section{چالش‌ها} \label{chap1:challenge}
\subsection{عدم توجه به فضای نهان} \label{chap1:latent_ignore}

\section{ساختار پایان‌نامه}




