\thispagestyle{empty}
\centerline{\textbf{\large{چکیده}}}
\begin{quote}

	علی رغم پیشرفت های چشم‌گیر
	در مساله سامانه‌های مکالمه، اما مکالمه‌گر‌های داده محور 
	به دلیل نبود مکالمات دانش‌محور در دادگان آموزشی و لحاظ نکردن دانش خارجی در معماری خود 
	اغلب قادر به ارائه دانش حاصل از دنیای واقعی و محتوامحور در بستر مکالمات خود نیستند. این پدیده باعث واردشدن خدشه به جنبه هوشمندی آن‌ها می‌باشد و استفاده از آن‌ها را در کاربردهای واقعی
	و دامنه‌باز
	با مشکل روبرو می‌کند.
	\newline  
	هدف از این پژوهش ارائه مدلی با استفاده از یادگیری ژرف جهت مکالمه مبتنی بر محتوای دانش و واقعیات بیرونی است؛ به ترتیبی که سامانه با استفاده از منابع دانش خارجی و قابل به روزرسانی قادر می‌شود که گفتگویی غنی از اطلاعات موجود در دانش بیرونی را انجام دهد. ذکر این نکته ضروری است که منابع دانش خارجی در این پژوهش به صورت مجموعه‌ای از اسناد متنی در نظر گرفته شده‌اند.
	\newline
	از آن جایی که در سالیان اخیر مدل‌های برآمده از شبکه‌های از پیش آموزش داده شده توانسته‌اند پرچمدار مسائل مختلف در حوزه پردازش زبان باشند، در این تحقیق نیز سعی شده است تا رویکرد حل مسئله مبتنی بر استفاده از قدرت این شبکه‌ها و یادگیری انتقالی باشد. آزمایش‌های این پژوهش  نیز نشان‌دهنده برتری روش‌های ارائه شده متکی بر شبکه‌های از پیش آموزش داده شده در مقابل روش‌های پیشین است.
	به طوری که با رویکرد تنظیم شبکه‌های 
	از پیش آموزش داده شده برت و بارت،
	 معیارهای فراخوانی در مساله انتخاب دانش از 
	$23.7$
	 به 
	 $76.52$
	  و امتیاز 
	\lr{F1}
	در مساله تولید پاسخ از 
	$32.2$
	 به 
	 $41.3$
	  افزایش و بهبود یافته‌اند. همچنین در این پژوهش، معیار جدیدی جهت سنجیدن کیفیت پاسخ‌های تولید شده توسط گپ‌زن استفاده 
	  شده
	  است که به علت در نظر گرفتن بازنمایی‌های معنایی جایگزین مناسبی برای معیارهایی نظیر 
	  \lr{F1}
	  است. 
	
	\vskip 1cm
	\textbf{کلمات کلیدی:} \textiranic{
		یادگیری ژرف،
		سامانه مکالمه،
		روبات‌ گپ‌زن دامنه‌باز،
		روبات گپ‌زن مبتنی بر دانش،
		شبکه‌های از پیش آموزش دیده
	}
\end{quote}