\chapter{جمع‌بندی و پیشنهادات}\label{Chap6}

در این پژوهش سعی شد تا ضمن بررسی مساله کلی مکالمه و نشان دادن اهمیت دانش بنیان کردن روبات‌های گپ‌زن،‌ مدلی جهت حل این مساله نیز ارائه شود. 
در ابتدا روش‌های رایج برای مدل کردن وظیفه مکالمه و همچنین راه‌حل‌های ارائه شده برای خلق گپ‌زن‌های دانش بنیان شرح داده شدند. در ادامه دو انقلاب بزرگ سالیان اخیر در حوزه پردازش زبان، یعنی معماری ترنسفورمر‌ها و نهضت انتقال یادگیری در پردازش زبان ارائه شده و مورد بحث واقع شدند. در کنار دو مطلب مذکور، معیار به تازه ابداع شده امتیاز برت نیز به عنوان سنجه‌ای مناسب در مسائل تولید متنی معرفی شد. 

در ادامه، مسئله گپ‌زن دانش بنیان به سه زیرمساله استخراج اسناد،‌ انتخاب دانش و تولید پاسخ خرد شد و تمرکز این پژوهش بر حل دو زیر مساله انتخاب دانش و تولید پاسخ معطوف گشت. مساله انتخاب دانش معادل با آموزش مدلی که بتواند فاصله تاریخچه گفتگو و هر جمله را تخمین بزند در نظر گرفته شد. این مدل از طریق تنظیم شبکه‌های از پیش آموزش یافته و البته عصاره گرفته شده برت بر روی مساله رتبه‌بندی حاصل شد و توانست پیشرفت خیره‌کننده ای نسبت به مدل پایه را به نمایش بگذارد. 

پس از حل چالش انتخاب دانش، برای حل زیرمساله تولید پاسخ دو معماری دنباله به دنباله
مبتنی بر شبکه‌های برت و بارت ارائه شدند. در معماری اول شبکه برت به عنوان رمزگذار و البته با اعمال تغییراتی به عنوان رمزگشا مورد تنظیم واقع شد. در معماری دوم، مدل از پیش آموزش یافته دنباله به دنباله بارت بر روی مساله تولید پاسخ دانش مجور تنظیم شد. بارت که پیش از این در وظایف خلاصه‌سازی از خود عملکرد مثبتی نشان داده بود در این جا نیز توانست موفقیت خودش را نسبت به مدل پایه تکرار کند. در ادامه آزمایشاتی نیز جهت روشن‌کردن تاثیر بعضی از راهکار‌های جانبی نظیر پیش‌آموزش مدل و استفاده از جستجوی پرتو بر تولید متن نیز انجام شده و نتایج‌های آن‌ها مورد بحث و بررسی قرار گرفتند. 

این پژوهش را می‌توان تلاشی اولیه و سطحی بر یک مساله نو و البته مهم و کاربردی دانست. جهت نیل کامل به هدف خلق یک گپ‌زن دانش بنیان، بسیاری از مسائل و چالش‌‌های حل‌نشده مربوط به این مساله بایستی مورد توجه و پژوهش قرار گیرند. بعضی از چالش‌های اساسی در مسیر خلق هر چه بهتر گپ‌زن دانش بنیان عبارتند از:

\begin{itemize}
	\item 
	\textbf{مشکل دادگان}
	دادگان جادوگر ویکی پدیا را بایستی اولین تلاش موثر برای ارائه دادگان آموزشی برای مسئله گپ‌زن دانش بنیان دانست. با این وجود،‌ نیاز به ایجاد دادگان آموزشی با موقعیت‌ها و سناریو‌های مختلف نسبت به جادوگر ویکی پدیا و البته با حجم بیشتر احساس می‌شود. البته با توجه به حالت مسئله مکالمه،
	 طبیعی است که جمع‌آوری یک دادگان مخصوص این تسک (به خصوص برای گپ‌زن دانش بنیان) نیازمند صرف هزینه زمانی و مالی و انسانی بالایی است. با در نظر گرفتن این شرایط دشوار،
	 یکی از راه‌حل‌های جالب توجه برای حل مشکل جمع‌آوری دادگان می‌تواند استفاده از شگرد 
	 \trans{افزون‌سازی داده}{data augmentation}
	 در نظر گرفته شود. راه حل دیگر نیز می‌تواند استفاده از تعداد دادگان زیاد حاضر در مساله پرسش و پاسخ و ساخت دادگان مکالمه‌ای دانش‌محور از روی آن‌ها باشد.
	\item
	\textbf{مدل‌سازی بهتر مکالمه}
	با وجود عملکرد قابل توجه مدل‌های فعلی مکالمه‌گر هوشمند (که از دیدگاه یادگیری با نظارت به عنوان رویکرد حل مساله استفاده می‌کنند) ؛ اما شکی نیست که مدل‌سازی مکالمه تا امروز به خوبی صورت نگرفته است. چه ‌آن که مکالمه مساله‌ای است که در آن به اصطلاح پدیده 
	\trans{پاداش با تاخیر}{delayed reward}
	وحود دارد و تولید یک پاسخ می‌تواند تاثیر خود را در چند نوبت بعدتر نشان دهد. 
	از این رو می‌توان ادعا کرد که مدل‌سازی کامل وظیفه مکالمه تنها با رویکرد 
	\trans{یادگیری تقویتی}{Reinforcement Learning}
	قابل دستیابی است. البته مدل‌سازی وظیفه مکالمه با رویکرد یادگیری تقویت خود در گرو پیشرفت این رویکرد در زمینه مدل‌سازی تولید متنی است.
	\item
	\textbf{پیش‌آموزش ترنسفورمر‌های دو پشته‌ای}
	با وجود درخشش مدل‌های از پیش آموزش داده شده ترنسفورمری تک پشته‌ای بر روی طیف وسیعی از مسائل پردازش زبانی، اما قدرت این مدل‌ها در مسائل دنباله به دنباله با ضعف جدی روبرو است. همانگونه که در این پژوهش دیده شد،‌ مدل‌های ترنسفورمری دو پشته‌ای پیش آموزش دیده نظیر بارت، در مسیر حل مسائل دنباله به دنباله بسیار امیدوارکننده می‌توانند عمل کنند. پیشنهاد می‌شود تا در طراحی معماری شبکه‌های از پیش آموزش دیده دوباره به ذات معماری ترنسفورمر که دارای پشته‌های رمزگذار و رمزگشا است، رجوع شود و روش‌ها و رویکرد آموزش متفاوتی نسبت به آن چه که در برت مشاهده شد مورد آزمایش قرار گیرند. در این صورت می‌توان امید داشت که این شبکه‌های دو پشته‌ای مسائل دنباله به دنباله نظیر مکالمه را به پیشرفت بیشتری برسانند.
\end{itemize}


