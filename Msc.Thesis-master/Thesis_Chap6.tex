\chapter{جمع‌بندی و پیشنهادات}\label{Chap6}

در این پژوهش سعی شد تا ضمن بررسی مساله کلی مکالمه و نشان دادن اهمیت دانش بنیان کردن روبات‌های گپ‌زن،‌ مدلی جهت حل این مساله نیز ارائه شود. 
در ابتدا روش‌های رایج برای مدل کردن وظیفه مکالمه و همچنین راه‌حل‌های ارائه شده برای خلق گپ‌زن‌های دانش بنیان شرح داده شدند. در ادامه دو انقلاب بزرگ سالیان اخیر در حوزه پردازش زبان، یعنی معماری ترنسفورمر‌ها و نهضت انتقال یادگیری در پردازش زبان ارائه شده و مورد بحث واقع شدند. در کنار دو مطلب مذکور، معیار به تازه ابداع شده امتیاز برت نیز به عنوان سنجه‌ای مناسب در مسائل تولید متنی معرفی شد. 

در ادامه، مسئله گپ‌زن دانش بنیان به سه زیرمساله استخراج اسناد،‌ انتخاب دانش و تولید پاسخ خرد شد و تمرکز این پژوهش بر حل دو زیر مساله انتخاب دانش و تولید پاسخ معطوف گشت. مساله انتخاب دانش معادل با آموزش مدلی که بتواند فاصله تاریخچه گفتگو و هر جمله را تخمین بزند در نظر گرفته شد. این مدل از طریق تنظیم شبکه‌های از پیش آموزش یافته و البته عصاره گرفته شده برت بر روی مساله رتبه‌بندی حاصل شد و توانست پیشرفت خیره‌کننده ای نسبت به مدل پایه را به نمایش بگذارد. 

پس از حل چالش انتخاب دانش، برای حل زیرمساله تولید پاسخ دو معماری دنباله به دنباله
مبتنی بر شبکه‌های برت و بارت ارائه شدند. در معماری اول شبکه برت به عنوان رمزگذار و البته با اعمال تغییراتی به عنوان رمزگشا مورد تنظیم واقع شد. در معماری دوم، مدل از پیش آموزش یافته دنباله به دنباله بارت بر روی مساله تولید پاسخ دانش مجور تنظیم شد. بارت که پیش از این در وظایف خلاصه‌سازی از خود عملکرد مثبتی نشان داده بود در این جا نیز توانست موفقیت خودش را نسبت به مدل پایه تکرار کند. در ادامه آزمایشاتی نیز جهت روشن‌کردن تاثیر بعضی از راهکار‌های جانبی نظیر پیش‌آموزش مدل و استفاده از جستجوی پرتو بر تولید متن نیز انجام شده و نتایج‌های آن‌ها مورد بحث و بررسی قرار گرفتند. 